\section{Esempi di tecniche di Geolocalizzazione}
Una delle tecniche proposte per risolvere il problema della localizzazione
indoor è \textbf{Active badge}: sviluppata nei prima anni novanta dalla
Olivetti, prevede un badge attivo e dei sensori; ogni 15 minuti il badge
trasmette un segale infrarossi. Il meccanismo di telemetria prevede di utilizzare
un computer, collegato ad un'unica rete di 128 sensori; ogni sensore ha una coda
FIFO per mantenere in un buffer informazioni su 20 badges.
\bigbreak
\textbf{Active floor} è una tecnologia, basata su piastrelle a pressione: a
seconda della piastrella su cui è rilevata la pressione è possibile localizzare
l'utente.
\bigbreak
Una successiva tecnologia di localizzazione indoor, decisamente più efficiente è
\textbf{Active bat}: è presente all'interno del soffitto una rete di sensori a
ultrasuoni; l'utente porta come wearable un piccolo emettitore: il segnale è
percepito dai sensori, i quali mandano le informazioni sul time of flight del
segnale infrarosso ad una base station, che calcola la posizione: con la
trilaterazione è possibile individuare l'utente con una precisione di 3
centimetri.
La sua evoluzione diretta è Ubisense.
\bigbreak
Queste tecniche presentano limitazioni in termini di privacy: chiunque possa
accedere alla base station, è in grado di conoscere la posizione di tutti gli
utenti del sistema. \textbf{Cricket system} rappresenta una tecnologia simile,
in cui è il dispositivo trasportabile a calcolare direttamente la posizione: una
maggior privacy è pagata con la necessità di avere dispositivi più intelligenti
e di conseguenza più costosi.
\bigbreak
\textbf{RADAR} rappresenta una tecnologia basata su RF di localizzazione indoor:
prevede di usare la struttura WiFi esistente. Si basa sul concetto di location
fingerprinting. È di tipo RSS empirico. Il principale vantaggio è che non è
necessaria una nuova infrastruttura.
\bigbreak
\textbf{placeLab} è una tecnologia che non richiede dei beacon specifici, ma
basandosi sul Wifi, Bluetooth o rete cellulare disponibili nell'ambiente
circostante, è in grado di determinare la posizione degli utenti. In questo modo
si può risolvere sia il problema dell'hardware dedicato per i sistemi di
positioning che quello della calibrazione dei componenti.
Naturalmente placelab fornisce risultati molto diversi a seconda che si basi sul
WiFi o sulla rete cellulare, sia in termini di precisione che di copertura.
Essendo basato sulla quantità di dispositivi nell'ambiente circostante inoltre
presenta prestazioni molto diverse a seconda che sia applicato in contesto
urbano, residenziale o suburbano.
Permette però di garantire un sistema di localizzazione che non richieda
all'utente di imparare ad usarlo e basato su beacons già disponibili.

\chapter{Mobile Sensing}
Dal concetto di ``strumentare l'ambiente'? con sensori e dispositivi, si passati a
strumentare le persone (accelerometri sul corpo o dispositivi come nel active
bat). Ad oggi però la tecnologia spinge sul dotare dispositivi general purpose,
come i telefoni cellulari, di sensori: sensori nati con scopi ludici, come il
giroscopio, o per attività quotidiane, come il microfono o l'accelerometro per
la rotazione, sono diventati presto la base per percepire l'ambiente
circostante.

All'inizio si strumentava l'ambiente, poi si è passato a strumentare le persone,
poi con i cellulari abbiamo general purpose machine che permettono di sentire le
cose. Inizialmente l'accelerometro era usato per la rotazione, il giroscopio per
i giochi e il microfono per le cose normali.

È stato necessario sviluppare tecniche molto fantasiose per superare le
limitazioni poste da piattaforme ideate per altri scopi, anche se i produttori
ad oggi stanno spingendo verso piattaforme hardware che facilitino il sensing e
framework OS che incorporino middleware general purpose per il sensing.

Le reti di sensori erano adatte a percepire l'ambiente, con hardware
specializzato per monitorare fenomeni specifici, le risorse interamente dedicate
al sensing, ma con un alto costo in termini di deployment e manutenzione.
Il sensing basato sui cellulari è molto comodo per le attività umane, basato
però su un hardware general purpose, non adatto a misurazioni accurate dei vari
fenomeni. Inoltre le risorse del sistema operativo vengono condivise tra le
diverse applicazioni. Il vantaggio massimo però è legato al bassissimo costo di
sviluppo e manutenzione, anche se non c'è la certezza che l'utente mantenga sul
proprio telefono l'applicazione.

Alcuni esempi di applicazioni sono:
\begin{itemize}
\item Attività fisica basata su giroscopio, bussola e accelerometro.
\item trasporti: sfruttando sensori come l'accelerometro, il GPS o il WiFi si
possono implementare meccanismi per il trasporto efficiente.
\item contesto e ambiente: mediante microfoni e videocamera si possono
implementare applicazioni come diari automatici o per app per la salute.
\item applicazioni per l'analisi della voce e delle conversazioni: con il
microfono si possono studiare applicazioni finalizzate allo studio sello stress
e delle interazioni sociali.
\end{itemize}

Tipicamente, il phone sensing segue un design pattern:
\begin{itemize}
\item collezionare dati raw, utilizzando i sensori
\item utilizzare i valori dei sensori per per inferire una particolare
attività:
\begin{itemize}
\item attività fisica: l'utente sta correndo?
\item context detection: ci sono altre persone nell'ambiente?
\end{itemize}
\item mostrare i risultati di alto livello all'utente o usarli per adattare il
comportamento dell'applicazione
\end{itemize}

Lo scopo di un cellulare è supportare molte applicazioni, ma il sensing è
un'attività estremamente intensiva dal punto delle risorse. Serve perciò
equilibrio tra le risorse necessarie per operare e l'accuratezza della detection
ottenuta.

\section{Sensing}
Esistono sostanzialmente due strategie di sensing: continua e a duty cycling.
Nel caso della strategia continua, il dispositivo, in seguito alla percezione di
un fenomeno, comincia a monitorare in maniera continua l'ambiente, ricavando
anche informazioni superflue e duplicate: i dati sono estremamente precisi, ma è
una metodologia estremamente costosa in termini di utilizzo della CPU e della
batteria, con una possibile riduzione della durata della batteria (fino a 6 ore
in stand-by); per tale ragione è usata su quei sensori più "economici", come
l'accelerometro.
Per quanto riguarda la strategia duty cycling, periodicamente il sensore si
attiva, percependo l'ambiente circostante: si alternano fasi di attività a fasi
di riposo. L'impatto sulla batteria è molto ridotto, ma a costo di misurazioni
meno precise e senza tenere conto che durante la fase di sleeping si potrebbero
avere degli eventi interessanti.
Una strategia per migliorare questa tecnica è quella di adattare il periodo di
sleeping sulla base degli eventi: se si ha percezione di eventi interessanti, il
periodo di riposo è ridotto in termini di durata, se invece non accade niente
nella finestra di percezione, allora si aumenta il tempo di sleeping.
Gli approcci tipici possono prevedere un incremento lineare con decrescita e
esponenziale o viceversa (dipende dall'evento). ottenendo un consumo di energia
migliore dell'approccio a sensing continuo e accuratezza più elevata del duty
cycling.
Il problema è che richiesta una conoscenza molto approfondita del contesto di
applicazione (ad esempio durante una conversazione, se si percepisce una nuova
voce, probabilmente la voce continuerà a parlare per del tempo).

\section{Inferenza}
Si chiama inferenza il processo di mappatura dei dati di basso livello a eventi 
significativi di alto livello.
la pipeline di inferenza comprende tre fasi: sensing, feature extraction e
classification.

Il motore inferenziale tipicamente deve essere disegnato basandoci sulle
seguenti fasi:
\begin{itemize}
\item collezionare i dati raw e etichettarli con informazioni reali.
\item il dataset tipicamente dovrebbe comprendere anche situazioni che non
stiamo cercando di individuare, ma che sono simili (ad esempio se vogliamo
individuare la camminata, dovremmo disporre di dati per la corsa e per lo stare
in piedi).
\item allenare il motore inferenziale con i dati collezionati
\item applicare il motore inferenziale all'applicazione target.
\end{itemize}

\section{Feature Extraction}
Questa fase prevede di identificare le features in un dataset che possono essere
usate per inferire un particolare tipo di attività.
L'insieme delle features selezionate dipende dal tipo di sensori e di attività
che si deve individuare. Il processo di design tipicamente include anche analisi
offline del training set per identificare le features corrette per il
particolare motore inferenziale in genere questo tipo di analisi è fatta
mediante un processo iterativo, in cui vengono testate features diverse.
Un possibile esempio è quello dell'individuazione della conversazione: si può
applicare FFT (trasformata di Fourier) ai samples audio, e comparare i training
data che sono etichettati come "conversazione" e come ``rumore non
conversazione''. Si seleziona come features la media e la standard deviation
sulla potenza dell'FFT. A questo punto anche una semplice linea come treshold
può fornire risultati relativamente accurati (anche se con un alto numero di
falsi positivi).
Nel caso di attività fisica ci si può basare sui dati dell'accelerometro,
confrontando l'essere seduti, con l'essere in piedi, il camminare e il correre:
le features possono essere la media, la deviazione standard e il numero di
picchi.

\section{Classification}
Dopo la fase di feature extraction, abbiamo a nostra disposizione un vettore di
features: la classificazione compara il vettore di features con un insieme
predefinito di classi di alto livello. In genere il motore di classificazione si
basa su tecniche di machine learning e è allenato a partire dai dati di training
etichettati. Tra i più comuni algoritmi di classificazione rientrano il Naive
Bayes, l'albero di decisione$\ldots$
\bigbreak
Dato un insieme delle features $F_1, \ldots, F_n$e un classificatore C, si stima la
probabilità di:
\begin{equation}
p(C|F_1, \ldots, F_n)
\end{equation}
che può essere approssimata da
\begin{equation}
p(C|F_1,\ldots,F_n)=\frac{1}{Z}p(C)\prod_{i=1}^{n}p(F_i|C)
\end{equation}
dove Z è una costante e può essere ignorata durante la comparazione. Usando il
training set stimiamo $p(F_i|C)$ e nel runtime, dato un insieme di valori per le
features $f_1, \ldots, f_n$, selezioniamo il classificatore che massimizza
\begin{equation}
p(C=c)\prod_{i=1}^{n}p(F_i=f_i|C=c)
\end{equation}

\section{Ottimizzazioni dell'inferenza}
Il sampling adattivo può decrementare il costo energetico, ma anche l'inferenza
può essere costosa (un esempio di attività estremamente costosa è la
rilevazione del parlato). Un altro aspetto chiave è il tempo di computazione.
Spesso conviene scaricare parte del costo energetico da altre parti, ad esempio
in un cloud.
Le sfide principali sono relative al bilanciare l'energia computazionale contro
il traffico di rete e bilanciare il tempo di attesa locale con quello remoto.
Mandare i dati raw dei sensori può essere più costoso in termini di network
energy, di quanto non venga risparmiato. La soluzione è quella di suddividere la
computazione: estrarre le features sul telefono e eseguire la classificazione
nel cloud. Tecniche più moderne e avanzate prevedono di distribuire il carico
computazionale adattivamente: decidere il miglior posto per eseguire la
computazione in maniera dinamica, stimando il costo di questa operazione in tempo
reale.
