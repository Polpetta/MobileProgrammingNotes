\documentclass[12pt,                    % corpo del font principale
               a4paper,                 % carta A4
               twoside,                 % impagina per fronte-retro
               openright,               % inizio capitoli a destra
               english
               ]{book}


% codifica di input; anche [latin1] va bene
% NOTA BENE! va accordata con le preferenze dell'editor
\usepackage[utf8]{inputenc}

%**************************************************************
% Importazione package
%**************************************************************

%\usepackage{amsmath,amssymb,amsthm}    % matematica


% per scrivere in italiano e in inglese;
% l'ultima lingua (l'italiano) risulta predefinita
\usepackage[english, italian]{babel}

\usepackage{bookmark}                   % segnalibri

\usepackage{caption}                    % didascalie

\usepackage{chngpage,calc}              % centra il frontespizio

% gestisce automaticamente i caratteri (")
\usepackage{csquotes}

% pagine vuote senza testatina e piede di pagina
\usepackage{emptypage}

\usepackage{epigraph}					% per epigrafi

\usepackage{eurosym}                    % simbolo dell'euro

% codifica dei font:
% NOTA BENE! richiede una distribuzione *completa* di LaTeX
\usepackage[T1]{fontenc}

% rientra il primo paragrafo di ogni sezione
%\usepackage{indentfirst}

\usepackage{graphicx}                   % immagini

\usepackage{hyperref}                   % collegamenti ipertestuali


% margini ottimizzati per l'A4; rilegatura di 5 mm
\usepackage[binding=5mm]{layaureo}

\usepackage{listings}                   % codici

\usepackage{microtype}                  % microtipografia

\usepackage{mparhack,relsize}  % finezze tipografiche

\usepackage{nameref}                    % visualizza nome dei riferimenti

\usepackage[font=small]{quoting}        % citazioni

\usepackage{subfig}                     % sottofigure, sottotabelle

\usepackage[italian]{varioref}          % riferimenti completi della pagina

\usepackage[dvipsnames]{xcolor}         % colori

\usepackage{booktabs}                   % tabelle
\usepackage{tabularx}                   % tabelle di larghezza prefissata
\usepackage{longtable}                  % tabelle su più pagine
% tabelle su più pagine e adattabili in larghezza
\usepackage{ltxtable}

% glossario
% per includerlo nel documento bisogna:
% 1. compilare una prima volta tesi.tex;
% 2. eseguire: makeindex -s tesi.ist -t tesi.glg -o tesi.gls tesi.glo
% 3. eseguire: makeindex -s tesi.ist -t tesi.alg -o tesi.acr tesi.acn
% 4. compilare due volte tesi.tex.
\usepackage[toc, acronym]{glossaries}

% permette di inserire le immagini/tabelle esattamente dove viene usato il
% comando \begin{figure}[H] ... \end{figure}
% evitando che venga spostato in automatico
\usepackage{float}

\usepackage{listings}
