\chapter{Web of Things}
\label{web-of-things}

\section{Web of Things vs Internet of Things}

\textbf{Internet of Things: una definizione} \\

Internet of Things è un sistema di oggetti fisici che possono essere
rilevati, monitorati, controllati o con cui si può interagire con
devices elettronici, che comunicano fra loro attraverso interfacce di rete
e, infine, che possono collegarsi a Internet.

C'è una nuova classe di oggetti che sta per entrare nelle nostre case:
le \textbf{Smart Things}. Una Smart Thing è un oggetto fisico ``digitalmente
aumentato'' con una o più delle seguenti caratteristiche:

\begin{itemize}
  \item Sensori (temperatura, luce, movimento, ecc.)
  \item Attuatori (displays, suoni, etc.)
  \item Proprietà computazionali (possono eseguire programmi)
  \item Interfacce di comunicazione (wired or wireless)
\end{itemize}

\begin{figure}[H]
  \centering
  \includegraphics[scale=0.6]{iotland.png}
  \caption{Scenario IoT. IoT è una rete di oggetti collegati ad Internet.}
  \label{fig:iotland}
\end{figure}

Il nome ``Internet delle cose'' significa semplicemente che un oggetto (e i
suoi servizi o dati da/verso esso) può essere acceduto e processato da altre
applicazioni attraverso l'esistente infrastruttura di Internet.

Le limitazioni dell'IoT diventano evidenti non appena si vuole integrare
dispositivi di vari produttori in una singola applicazione o sistema.

\begin{figure}[H]
  \centering
  \includegraphics[scale=0.5]{integration_problem.png}
  \caption{Nell'IoT, oggi esistono centinaia di protocolli incompatibili.
Ciò rende l'integrazione di dati e servizi da vari dispositivi estremamente
complessa e costosa.}
  \label{fig:integration_problem}
\end{figure}

\subsection{Sfuggire al pattern ``un device, un protocollo, una sola app''}

Nel Web of Things (non IoT!), è possibile accedere a qualsiasi dispositivo
utilizzando protocolli web standard.

Collegare dispositivi eterogenei al Web rende la loro integrazione in sistemi
più ampi e/o applicazioni molto più semplice.

L'idea di applicare gli strumenti e le tecniche presenti nel
Web allo sviluppo di scenari IoT si chiama Web of Things.

Mentre IoT si occupa di risolvere problemi di rete, il Web of Things si basa
esclusivamente su protocolli e strumenti a livello di applicazione.

Mappare un qualsiasi dispositivo in un contesto Web rende agnostico il
Web of Things rispetto ai protocolli di livello fisico e di trasporto
utilizzati dai dispositivi.

\begin{figure}[H]
  \centering
  \includegraphics[scale=0.6]{iotstack.png}
  \caption{Il Web of Things riguarda solo il più alto livello OSI, che
gestisce applicazioni, servizi e dati.
Lavorare con un elevato livello di astrazione consente di collegare dati
e servizi da molti dispositivi indipendentemente dai veri e propri protocolli
di trasporto usati.
Al contrario, l'Internet of Things non sostiene un particolare protocollo a
livello di applicazione e di solito si concentra sugli strati inferiori dello
Stack OSI.}
  \label{fig:iotstack}
\end{figure}

\textbf{Lo scopo è quello di accedere alle cose in modo standardizzato e
trasparente.}

\begin{figure}[H]
  \centering
  \includegraphics[scale=0.7]{iotweb.png}
  \caption{Il Web of Things è la capacità di utilizzare gli standard web moderni
direttamente su dispositivi embedded.}
  \label{fig:iotweb}
\end{figure}

Questo rende il Web il substrato ideale per la creazione di un'architettura
``universale'' e di Application Programming Interface (API) con cui gli oggetti possono
interagire.

In pratica, questo significa che l'utente può iniziare a interagire con gli oggetti
via Web

I dati real-time raccolti da sensori distribuiti possono essere facilmente
recuperati, elaborati e visualizzati in pagine Web utilizzando HTML, CSS e
JavaScript.

\begin{figure}[H]
  \centering
  \includegraphics[scale=0.5]{iotrest.png}
  \caption{Un URL per ogni cosa e un'API RESTful.
Il Web delle cose consente agli sviluppatori e alle applicazioni di scambiare
dati con qualsiasi dispositivo che utilizza le richieste HTTP / WebSockets
standard, indipendentemente dal modo in cui il dispositivo è collegato.}
  \label{fig:iotrest}
\end{figure}

\textbf{MA} anche se i protocolli Web sono disponibili e utilizzabili per i
dispositivi IoT, possono essere troppo pesanti per alcune applicazioni di IoT.
Diamo un'occhiata ad altri protocolli.

\subsection{Non tutti usano HTTP}

\textbf{MQTT}\\

MQ Telemetry Transport (MQTT) è un protocollo open source per devices con
vincoli di banda e reti con alta latenza.

Ha un trasporto di messaggistica publish/subscribe che è estremamente leggero.

MQTT è efficiente in quanto a consumi di banda e data agnostic.
Aiuta a ridurre al minimo i requisiti di risorse per un dispositivo IoT,
pur tentando di garantire l'affidabilità del trasporto.

\textbf{Proprietà:}

\begin{itemize}
  \item modello client/server
  \item i clients si iscrivono ai canali in base a dei ``topic di interesse''
  \item i canali per i topic sono gerarchici (es. room2BC/heating)
  \item 3 livelli QoS: ``Fire and forget'',  ``delivered at least once'' e
 ``delivered exactly once''.
  \item autenticazione basata su username/password
  \item TCP over SSL/TLS
\end{itemize}

\textbf{CoAP}\\

Il Constrained Application Protocol (CoAP) è stato progettato per essere
utilizzato su reti vincolate e a basso consumo. CoAP è un protocollo RESTful.
È semanticamente allineato con HTTP e ha persino un mapping uno-a-uno da e
verso HTTP.\\

\textbf{Proprietà:}

\begin{itemize}
  \item i pacchetti sono molto più piccoli rispetto a HTTP TCP.
  \item (simpler and faster to parse with small memory footprint) ?
  \item In UDP, interagisce con HTTP e il web RESTful attraverso semplici
proxies
  \item modello client/server, in cui i clients possono chiamare GET, PUT,
POST and DELETE per ottenere/modificare/inserire/cancellare risorse.
  \item I dispositivi CoAP in grado di supportare DTLS supportano RSA e AES o
ECC e AES
\end{itemize}

\begin{figure}[H]
  \centering
  \includegraphics[scale=0.4]{protocol_comparison.png}
  \caption{Sulla sinistra, lo stack di protocollo per applicazioni Web può
facilmente produrre un overhead di dati di centinaia o migliaia di byte.
A destra invece, i protocolli IoT sono ottimizzati per dispositivi e reti
vincolati, e producono un overhead di dati molto più piccolo (decine di
byte).}
  \label{fig:protocol_comparison}
\end{figure}

\subsection{Integration Patterns}

\textbf{Comunicazione diretta}

Nel caso più semplice, una Web Thing è semplicemente una API alla quale i
clients inviano richieste.

Il Client e la Web Thing possono essere sulla stessa rete o meno.

\begin{figure}[H]
  \centering
  \includegraphics[scale=0.5]{direct_communication.png}
  \caption{Comunicazione diretta tra un client e una web thing}
  \label{fig:direct_communication}
\end{figure}

\textbf{Gateway}

La connettività/comunicazione basata su gateway viene utilizzata quando un oggetto
non può offrire direttamente un'API Web.

In questo caso una Web Thing intermedia - il gateway - espone un'API Web
per conto dell'oggetto.

La Web Thing quindi funge da proxy per l'oggetto (o da gateway a seconda della
complessità della traduzione).

\begin{figure}[H]
  \centering
  \includegraphics[scale=0.5]{gateway.png}
  \caption{Connettività basata su gateway tra un client e una web thing}
  \label{fig:gateway}
\end{figure}

\textbf{Cloud}

Il terzo caso è simile al precedente ma questa volta il gateway è un servizio
cloud.

\begin{figure}[H]
  \centering
  \includegraphics[scale=0.5]{cloud.png}
  \caption{Connettività basata su cloud tra un client e una web thing}
  \label{fig:cloud}
\end{figure}

\subsection{Web of Things Architecture}

While there are ongoing efforts to standardise it, the Web of Things is a set 
of best practices that can be classified according to the Web of Things
architecture.
The architecture proposes four main layers that are used as a framework to
classify the different patterns and protocols involved.

\begin{figure}[H]
  \centering
  \includegraphics[scale=0.5]{wotarchitecture.png}
  \caption{Web of Things Architecture}
  \label{fig:wotarchitecture}
\end{figure}

\textbf{Accessibility layer}\\

This layer deals with the access of things to the
Internet and ensure they expose their services via Web APIs
(already discussed in the previous sections of this chapter).\\

\textbf{Findability layer}\\

Marking things accessible via an HTTP and WebSocket API is great but it doesn't
mean applications can really  ``understand'' what the Thing is, what data or
services it offers, and so on.

This is where the second layer – Find – becomes interesting.
This layer ensures that your Thing can not only be easily used by other HTTP
clients but can also be findable and automatically usable by other WoT
applications.

The approach here is to reuse web semantic standards to describe things and
their services.

This enables searching for things through search engines and other web indexes
as well as the automatic generation of user interfaces or tools to interact
with Things.\\

\textbf{Sharing layer}\\

The Internet of Things will only blossom (fiorirà) if Things have a way to
securely share data across services.

This is the responsibility of the Share layer, which specifies how the data
generated by Things can be shared in an efficient and secure manner over
the web.

At this level, another batch of Web protocols help.
First, TLS, the protocol that makes transactions on the Web secure.
Then, techniques such as delegated web authentication mechanisms like OAuth
which can be integrated to our Things' APIs. Finally, we can also use social
networks to share Things and their resources to create a Social Web of
Things.\\

\textbf{Composition layer}\\

Finally, once Things are on the Web (layer 1) where they can be found by humans
and machines (layer 2) and their resources can be shared securely with others
(layer 3), it's time to look at how to build large-scale, meaningful
applications for the Web of Things. In other words, we need to understand the
integration of data and services from heterogeneous Things into an immense
ecosystem of web tools such as analytics software and mashup platforms.
Web tools at the Compose layer range from web toolkits - for example,
JavaScript SDKs offering higher-level abstractions - to dashboards with
programmable widgets, and finally to physical mashup tools such as Node-RED as
shown below.

\begin{figure}[H]
  \centering
  \includegraphics[scale=0.4]{nodered.png}
  \caption{A Physical Mashup with Node-RED;
Monitor a process / sensor;
Automatic control of heating system;}
  \label{fig:nodered}
\end{figure}

\subsubsection{Conclusioni}

Web of Things è un protocollo di applicazioni ad alto livello progettato per
massimizzare l'interoperabilità nello IoT.

Le tecnologie Web sono ampiamente diffuse e offrono tutta la flessibilità e le
caratteristiche necessarie per le future applicazioni IoT. \\

\textbf{Fonti:}\\

Un mix di:

\begin{itemize}
  \item Course slides
  \item \url{http://webofthings.org/2016/01/23/wot-vs-iot-12/}
  \item \url{http://webofthings.org/2017/04/08/what-is-the-web-of-things/}
  \item \url{https://www.micrium.com/iot/internet-protocols/}
  \item \url{https://www.w3.org/blog/wotig/}
\end{itemize}
