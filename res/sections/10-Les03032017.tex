\chapter{Introduction to mobile computing}

\section{Historical overview of computing}
The technology is continuously evolving. In 2007 mobile phones start to be able 
to access the internet.
What is mobile computing? It's performing a task while moving. Now there is a 
lot of flexibility and mobility thanks to this. Moreover, there are scenarios 
where there is not mobile connectivity or there is no coverage. Similar, there 
are remote areas, like the last miles where there isn't wired connection, and 
mobile connectivity is the answer to this problem.

\subsection{Why go mobile?}
As already said above, mobile connections are very useful in remote areas, and 
also in rescue situations where there is no time to set up a proper network.

There are mobile's degree:
\begin{itemize}
  \item Physical mobility \\
  Devices visit and leave the space (carried by human agents or by robots)
  \item Logical mobility
  \begin{itemize}
    \item A software component (for example a Java applets)
    % Some stuff is missing :(
  \end{itemize}
\end{itemize}

Important evolution in mobile programming:
\begin{itemize}
  \item Miniaturization
  \item Connectivity
  \item Portability
  \item Convergence
  \item Divergence
  \item Applications
  \item Digital ecosystems
\end{itemize}

\section{Current trends and driving factors}

\section{Fundamental Challenges}

There are different challenges in the mobile word.
One of these is the resource constraints, where resources are few sometimes 
(for instance memory or in particular energy constraints).

Another important challenge is the difference between a wired environment (like 
a normal computer) versus a wireless environment, where you have continuous 
disconnections, low bandwidth and an high variable delay in communications. 
Huge problems comes also with transfer protocols.

\section{Toward ubiquitous computing, challenges and opportunities}

\subsection{ParcTab}

The ParcTab was one of the first mobile devices. Instead of using wireless 
communication it used infrared communications. It used RFID technology to 
acknowledge his position, and based on the room in the office, a particular 
application was triggered (for example when inside a conference room a shell 
was triggered automatically to start a presentation). This was a first example 
of a context awareness device.