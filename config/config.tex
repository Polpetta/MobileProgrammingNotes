%**************************************************************
% file contenente le impostazioni della tesi
%**************************************************************

%**************************************************************
% Frontespizio
%**************************************************************
\newcommand{\myName}{Davide Polonio, Giacomo Manzoli, Matteo Di Pirro, Marco
Zanella, Emanuele Carraro, Luca Pajola, Davide Orsini, Eduard Bicego, Federico
Tavella, Guglielmo Faggioli, Tommaso Padovan}           % autore
\newcommand{\myTitle}{Appunti di Mobile Programming}
\newcommand{\myUni}{Università degli Studi di Padova}           % università
\newcommand{\myFaculty}{Corso di Laurea Magistrale in Informatica}         %
%facoltà
\newcommand{\myDepartment}{Dipartimento di Matematica}          % dipartimento
\newcommand{\myLocation}{Padova}                                % dove
\newcommand{\myAA}{2016-2017}                                   % anno
\newcommand{\myCopyright}{CC-BY-SA-4.0}                         % copyright
\newcommand{\myVersion}{v0.4}                                   % versione
\newcommand{\myRelease}{
https://github.com/Polpetta/MobileProgrammingNotes/releases}
\newcommand{\myIssue}{https://github.com/Polpetta/MobileProgrammingNotes/issues}
\newcommand{\myPullRequest}{
https://github.com/Polpetta/MobileProgrammingNotes/pulls}

%**************************************************************
% Impostazioni di impaginazione
% see: http://wwwcdf.pd.infn.it/AppuntiLinux/a2547.htm
%**************************************************************

\setlength{\parindent}{14pt}   % larghezza rientro della prima riga
\setlength{\parskip}{0pt}   % distanza tra i paragrafi


%**************************************************************
% Impostazioni di caption
%**************************************************************
\captionsetup{
    tableposition=top,
    figureposition=bottom,
    font=small,
    format=hang,
    labelfont=bf
}

%**************************************************************
% Impostazioni di glossaries
%**************************************************************
\input{res/glossary} % database di termini
\makeglossaries


%**************************************************************
% Impostazioni di graphicx
%**************************************************************
\graphicspath{{res/img/}} % cartella dove sono riposte le immagini


%**************************************************************
% Impostazioni di hyperref
%**************************************************************
\hypersetup{
    %hyperfootnotes=false,
    %pdfpagelabels,
    %draft,	% = elimina tutti i link (utile per stampe in bianco e nero)
    colorlinks=true,
    linktocpage=true,
    pdfstartpage=1,
    pdfstartview=FitV,
    % decommenta la riga seguente per avere link in nero (per esempio per la
%stampa in bianco e nero)
    %colorlinks=false, linktocpage=false, pdfborder={0 0 0}, pdfstartpage=1,
%pdfstartview=FitV,
    breaklinks=true,
    pdfpagemode=UseNone,
    pageanchor=true,
    pdfpagemode=UseOutlines,
    plainpages=false,
    bookmarksnumbered,
    bookmarksopen=true,
    bookmarksopenlevel=1,
    hypertexnames=true,
    pdfhighlight=/O,
    %nesting=true,
    %frenchlinks,
    urlcolor=webbrown,
    linkcolor=RoyalBlue,
    citecolor=webgreen,
    %pagecolor=RoyalBlue,
    %urlcolor=Black, linkcolor=Black, citecolor=Black, %pagecolor=Black,
    pdftitle={\myTitle},
    pdfauthor={\textcopyright\ \myName, \myUni, \myFaculty},
    pdfsubject={},
    pdfkeywords={},
    pdfcreator={pdfLaTeX},
    pdfproducer={LaTeX}
}

%**************************************************************
% Impostazioni di itemize
%**************************************************************
% \renewcommand{\labelitemi}{$\ast$}

%\renewcommand{\labelitemi}{$\bullet$}
%\renewcommand{\labelitemii}{$\cdot$}
%\renewcommand{\labelitemiii}{$\diamond$}
%\renewcommand{\labelitemiv}{$\ast$}


%**************************************************************
% Impostazioni di listings
%**************************************************************

\definecolor{codegreen}{rgb}{0,0.6,0}
\definecolor{codegray}{rgb}{0.5,0.5,0.5}
\definecolor{backcolor}{rgb}{0.98,0.98,0.98}
\lstset{
	backgroundcolor=\color{backcolor},
	commentstyle=\color{Peach}\ttfamily,
	keywordstyle=\color{RoyalBlue},
	numberstyle=\tiny\color{codegray},
	stringstyle=\color{SeaGreen}\ttfamily,
	basicstyle=\footnotesize\ttfamily,
	breakatwhitespace=false,
	breaklines=true,
	captionpos=b,
	keepspaces=true,
	numbers=left,
	numbersep=5pt,
	showspaces=false,
	showstringspaces=false,
	showtabs=false,
	tabsize=2,
	frame=trbl, % draw a frame at the top, right, left and bottom of the listing
	frameround=ftff, % angolo in basso a destro curvo
	framesep=4pt, % quarter circle size of the round corners,
	inputencoding=utf8,
	extendedchars=true,
	literate={á}{{\'a}}1 {à}{{\`a}}1 {é}{{\'e}}1 {è}{{\`e}}1 {ù}{{\`u}}1 {ò}{{\`o}}1 {ì}{{\`i}}1,
	belowskip=1em,
	aboveskip=1em,
}

\lstdefinelanguage{XML}
{
	% list of keywords
	morekeywords={application, uses-permission, uses, permission, xml, version, encoding, service, activity, intent -filter, manifest, fragment, LinearLayout, FrameLayout,
	application, service},
	morecomment=[s]{<!--}{-->}, % s is for start and end delimiter
	morestring=[b]" % defines that strings are enclosed in double quotes
}

\lstdefinelanguage{Java}
{
	morekeywords={ListView, ArrayAdapter, Context, R, String, WifiManager, Intent,
		String, public, private, List, while, null, if, return, Cursor, Fragment, Uri,
		static, final, int, case, break, CursorLoader, ContentValues, HeadlessFragment,
		protected, @Override, boolean, TextView, switch, SharedPreferences, void,
		View, for, extends, class, FragmentTransaction, WorkoutDetailFragment, long,
		throw, new, this, true, false, Button},
	morecomment=[l]{//},
	morecomment=[s]{/*}{*/},
	morestring=[b]"
}


%**************************************************************
% Impostazioni di xcolor
%**************************************************************
\definecolor{webgreen}{rgb}{0,.5,0}
\definecolor{webbrown}{rgb}{.6,0,0}


%**************************************************************
% Altro
%**************************************************************

\newcommand{\omissis}{[\dots\negthinspace]} % produce [...]

% eccezioni all'algoritmo di sillabazione
\hyphenation
{
    ma-cro-istru-zio-ne
    gi-ral-din
}

\newcommand{\sectionname}{sezione}
% \addto\captionsitalian{\renewcommand{\figurename}{figura}
%                        \renewcommand{\tablename}{tabella}}

\newcommand{\glsfirstoccur}{\ap{{[g]}}}

\newcommand{\intro}[1]{\emph{\textsf{#1}}}
